\documentclass[12pt, oneside]{memoir}

\nonzeroparskip
\setlength{\parindent}{0pt}

\newcommand{\var}[1]{\emph{#1}}
\newcommand{\cp}[1]{U+#1}

\begin{document}

\begin{titlingpage}
  \begin{vplace}
    \centering
      \rule{\textwidth}{1pt}
      {\LARGE THE VLINDER \\[0.5\baselineskip] LANGUAGE SPECIFICATION}
      \rule{\textwidth}{1pt}
      \vspace{5pt}

      {\large Version 0.0.1}

      \vfill
      rightfold
    \par
  \end{vplace}
\end{titlingpage}

\chapter{Source code acquisition}

The implementation acquires source code by following the following steps:

\begin{enumerate}
  \item The entire source file is read as a sequence of bytes from now on
        referred to as \var{source}.
  \item If \var{source} is not a UTF-8 encoding, the program is ill-formed.
  \item Every instance of code point sequence \cp{000D} \cp{000A} in
        \var{source} is replaced with \cp{000A}. The result of this operation
        is from now on referred to as \var{source}.
  \item If \var{source} contains a code point in one of the ranges
        [\cp{0000}, \cp{0009}] and [\cp{000B}, \cp{001F}], the program is
        ill-formed.
\end{enumerate}

The acquired source code is what is now referred to as \var{source}.

\chapter{Lexical syntax}

\section{Reserved words}

The following words are reserved and cannot be used as identifiers:

\begin{itemize}
  \item \texttt{false}
  \item \texttt{import}
  \item \texttt{PIC}
  \item \texttt{PICTURE}
  \item \texttt{struct}
  \item \texttt{sub}
  \item \texttt{true}
  \item \texttt{typealias}
  \item \texttt{union}
\end{itemize}

In addition, the reserved words \texttt{PIC} and \texttt{PICTURE} are
case-insensitive.

\chapter{Expressions}

\begin{verbatim}
<expr> = <call-expr>
\end{verbatim}

\section{Call expressions}

\begin{verbatim}
<call-expr> = <value-call-expr>
            | <type-call-expr>
            | <primary-expr>
\end{verbatim}

\subsection{Value call expressions}

\begin{verbatim}
<value-call-expr> = <call-expr> '(' <expr> %% ',' ')'
\end{verbatim}

\subsection{Type call expressions}

\begin{verbatim}
<type-call-expr> = <call-expr> '[' <type-expr> %% ',' ']'
\end{verbatim}

\section{Primary expressions}

\begin{verbatim}
<primary-expr> = <block-expr>
               | <name-expr>
               | <string-expr>
               | <struct-expr>
\end{verbatim}

\subsection{Block expressions}

\begin{verbatim}
<block-expr> = '{' <block-expr-stmt>* '}'
<block-expr-stmt> = <decl> | <expr> ';'
\end{verbatim}

\subsection{Boolean expressions}

\begin{verbatim}
<boolean-expr> = 'true' | 'false'
\end{verbatim}

\subsection{Name expressions}

\begin{verbatim}
<name-expr> = <name>
\end{verbatim}

\subsection{String expressions}

\begin{verbatim}
<string-expr> = <string>
\end{verbatim}

\subsection{Struct expressions}

\begin{verbatim}
<struct-expr> = <type-expr> '{' <struct-expr-field> %% ',' '}'
<struct-expr-field> = <identifier> ':' <expr>
\end{verbatim}

\end{document}
